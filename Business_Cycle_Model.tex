\documentclass{article}

\usepackage[utf8]{inputenc}
\usepackage[english]{babel}
\usepackage{amsthm}
\usepackage{amsmath}
\usepackage{graphicx}
\usepackage{hyperref}

\begin{document}


\hypertarget{topic-agent-based-computational-economics-simulation-model-of-the-capitalist-business-cycle.}{%
\section{Topic: Agent Based Computational Economics simulation model of
the capitalist business
cycle.}\label{topic-agent-based-computational-economics-simulation-model-of-the-capitalist-business-cycle.}}

Three papers:

\hypertarget{endogenous-credit-money-from-cash-flow}{%
\subsection{1. Endogenous Credit Money from Cash
Flow}\label{endogenous-credit-money-from-cash-flow}}

Simplest abstract schematic model of typical cash flows for large
numbers of \textbf{articulated, imbricated, ramified} and
\textbf{layered} production units and financial institutions with choice
from even larger numbers of more and less labor intensive techniques as
basis for endogenous credit money.

\hypertarget{emergent-cycles-from-disproportions-induced-by-gestation-period}{%
\subsection{2. Emergent cycles from disproportions induced by gestation
period}\label{emergent-cycles-from-disproportions-induced-by-gestation-period}}

Simplest abstract schematic model of disproportions between demand,
supply and prices of commodities produced for consumption by other
businesses and those for final consumption by workers and owners based
on gestation lag between orders based on current prices and subsequent
deliveries of products ordered with outputs from a ``higher'' layer not
matching inputs required by a ``lower'' layer. Technical change to less
labor intensive techniques accelerated after prices and average rate of
profit crash enables initial investment to kick off the next cycle.

\hypertarget{ramified-changing-technologies}{%
\subsection{3. Ramified Changing
Technologies}\label{ramified-changing-technologies}}

Emergence of new final and intermediate products by increasingly
specialized branching from existing techniques to define new techniques
with cheaper inputs or superior outputs.

\hypertarget{notes-for-paper-1}{%
\subsection{Notes for paper 1}\label{notes-for-paper-1}}

\hypertarget{liquidity}{%
\subsubsection{1.1 Liquidity}\label{liquidity}}

Initial study of practical cash flow forecasting and liquidity risk
management to extract the simplest essential features necessary for a
schematic model that includes endogenous credit money. Mathematical
description of time series for forecasts and yield curves etc is
expected to require tensors.

Starting text: Brian Coyle, ``Cash Flow Forecasting and Liquidity
Financial Risk Management''

\begin{enumerate}
\def\labelenumi{\arabic{enumi}.}
\tightlist
\item
  Cashflow-Forecasting-and-Liquidity-Risk-Management-Series-.chm
\end{enumerate}

Other \emph{starting} texts: TBD from /Lehman\_Bib/start/:

liquidity\_risk.md Liquidity\_Risk\_Management/ Valuation/

and other folders and other references not yet found

\begin{enumerate}
\def\labelenumi{\arabic{enumi}.}
\setcounter{enumi}{1}
\item
  Liquidity-Risk-Management-in-Banks-Economic-and-Regulatory-Issues.pdf
\item
  Modeling-Liquidity-Risk-With-Implications-for-Traditional-Market-Risk-Measurement-and-Management.pdf
\item
  An-Introduction-to-Banking-Liquidity-Risk-and-Asset-Liability-Management.pdf
\end{enumerate}

\hypertarget{articulation}{%
\subsubsection{1.2 Articulation}\label{articulation}}

Liquidity from stocks or cash and credit provides the essential
buffering and lubrication for flexible joints between articulated
division of social production among autonomous units operating
successive stages of production processes with inputs consumed by other
producing units and final consumers purchased as outputs from other
producing units and workers.

\hypertarget{imbrication}{%
\subsubsection{1.3 Imbrication}\label{imbrication}}

The overlapping pattern (like feathers or roof tiles) of production,
consumption, demand, supply, purchases, sales, credit and debt requires
a large scale Agent Based Model to simulate the essential features that
interact to produce both coherence growth with business cycles. Micro
foundations of macro model by autonomous agents with actual social
relations of production.

\hypertarget{ramified}{%
\subsubsection{1.4 Ramified}\label{ramified}}

The branching division of labour with larger numbers of inputs and
production stages as production shifts to less labor intensive
techniques needs to be included from the start even with initial choice
from only a large static list of existing known technologies. This
avoids creating unnecessary obstacles for later attempt at much more
difficult problem of actually including unpredictable discoveries in
paper 3. But also avoids getting bogged down in paper 3 before
delivering papers 1 and 2.

\hypertarget{abstraction}{%
\subsubsection{1.5 Abstraction}\label{abstraction}}

Capture only the essence by abstracting from everything that is not
essential for producing a model with cyclic growth so as to eventually
produce a working simulation model. Avoid all detail that would be
needed for greater realism and actual practical models including:

\begin{enumerate}
\def\labelenumi{\arabic{enumi}.}
\item
  Spatial distribution and non-homogeneity of resources, production
  units, transport links, workers, owners and governments. This
  eliminates such issues as currencies, rent, seasonality, quality,
  taste, fashion, skills, experience, propensities, inheritance politics
  etc that are either complicated or impossible to model.
\item
  Historical development of social relations of production. Assume just
  individuals with a distribution of portfolio wealth. The distribution
  can change as portfolios grow and shrink but the people don't change
  in the initial schematic model.
\item
  Changing technologies. Innovation still occurs by choosing to adopt
  different techniques from among the large static range available
  depending on current and expected future prices of inputs and outputs.
  But we defer actual invention until paper 3.
\item
  Realistic production and consumption functions. Sufficient to treat
  final consumption like production and use simplest possible
  technologies that can be handled by Mixed Integer Linear Programming
  techniques.
\end{enumerate}

This corresponds roughly to the extreme simplifications used to
illustrate neoclassical fables about non-cyclic growth with only the
minimum necessary additions to enable business cycles genuinely
enogenous to the micro foundations rather than imposed arbitrarily and
derived from finance rather than from production relations.

\end{document}
